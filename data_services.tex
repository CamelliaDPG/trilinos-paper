\todo{Describe Tpetra, idea of performance portability, Kokkos and Kokkos Kernels}

\todo{Do we want Teko here?}\textcolor{orange}{ LBV: Why not have Teko in the linear solver section?}

\subsection{Kokkos Kernels}\label{subsec:kk}
Kokkos Kernels is part of the Kokkos ecosystem and provides node local implementations
of mathematic kernels widely used across packages in Trilinos. As a member of the Kokkos
ecosystem, Kokkos Kernels is tightly integrated on Kokkos features and aims at
delivering performance portable algorithms across major CPU and GPU based HPC systems.
Due to its node local nature, Kokkos Kernels does not rely on MPI or other communication
library unlike numerous other packages in Trilinos.

The implementation of Kokkos Kernels algorithms leverage the hierarchical parallelism
exposed by the Kokkos library and increasingly provides coverage for stream callable
kernels. To ensure flexibility for the distributed libraries that might call its
algorithms, Kokkos Kernels provides thread safe and asynchronous implementations for
most its kernels. Kokkos Kernels also serves as a major point of integration for vendor
optimized libraries such as cuBLAS, cuSPARSE, rocBLAS, rocSPARSE, MKL, ARMpl and others.

The capabilities that Kokkos Kernels provides can be divided in four major categories:
1. BLAS algorithms, 2. sparse linear algebra and preconditioners, 3. graph algorithms
4. batched dense and sparse linear algebra. The main points of integration of Kokkos
Kernels in Trilinos are Tpetra for the dense and sparse linear algebra capabilities,
Ifpack2 for the preconditioners and batched algorithms, the multigrid package MueLu
that relies on these features both directly and indirectly as well as on some
specialized algorithms such as graph coloring/coarsening and fused Jacobi-SpGEMM kernels.

Similarly to the Kokkos library, Kokkos Kernels is developed in its own GitHub repository\footnote{https://github.com/kokkos/kokkos-kernels}
outside of the Trilinos github repository. Every version of the library is integrated
and tested in Trilinos as part of the Kokkos ecosystem release process.
