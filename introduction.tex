
\todo{Add introduction here. Add a subsection on Trilinos organization as a set of five products.}

Trilinos is a community-developed, open source software framework that facilitates building large-scale, complex, multiscale, multiphysics engineering and scientific problems. While Trilinos can run on small workstations to large supercomputers, the typical use of Trilinos is on the leadership class systems with new or emerging hardware architectures.

% History
Trilinos was originally conceived as framework of three packages for distributed memory systems. The original Trilinos publication~\cite{Heroux2005a} describes the motivation and the philosophy behind Trilinos and the capabilities that existed in Trilinos at that time. Trilinos today is similar to the Trilinos that was envisioned two decades ago in some aspects. However, Trilinos today is also very different in several other aspects. These changes were necessitated by the changes in programming models, application needs, hardware architectures, and algorithms. Trilinos has grown from a library of three packages to a library with more than fifty packages with functionality and features supporting wide range of applications.

% Purpose
This article is an attempt to capture a snapshot of where Trilinos is today as opposed to eighteen years ago when the original Trilinos article was written. We will focus on the major developments within Trilinos in the last decade, new features and functionality that has been added to enable scientific and engineering applications. This article will be an overview of the features and we refer to the extensive reference list for the details of these features. We are also cognizant of the fact that as a software that is actively developed this document will become outdated even before its publication. We would focus on the high level features and project that we expect to remain stable for several years.
