%%
%% This is file `sample-acmsmall.tex',
%% generated with the docstrip utility.
%%
%% The original source files were:
%%
%% samples.dtx  (with options: `acmsmall')
%% 
%% IMPORTANT NOTICE:
%% 
%% For the copyright see the source file.
%% 
%% Any modified versions of this file must be renamed
%% with new filenames distinct from sample-acmsmall.tex.
%% 
%% For distribution of the original source see the terms
%% for copying and modification in the file samples.dtx.
%% 
%% This generated file may be distributed as long as the
%% original source files, as listed above, are part of the
%% same distribution. (The sources need not necessarily be
%% in the same archive or directory.)
%%
%%
%% Commands for TeXCount
%TC:macro \cite [option:text,text]
%TC:macro \citep [option:text,text]
%TC:macro \citet [option:text,text]
%TC:envir table 0 1
%TC:envir table* 0 1
%TC:envir tabular [ignore] word
%TC:envir displaymath 0 word
%TC:envir math 0 word
%TC:envir comment 0 0
%%
%%
%% The first command in your LaTeX source must be the \documentclass
%% command.
%%
%% For submission and review of your manuscript please change the
%% command to \documentclass[manuscript, screen, review]{acmart}.
%%
%% When submitting camera ready or to TAPS, please change the command
%% to \documentclass[sigconf]{acmart} or whichever template is required
%% for your publication.
%%
%%
\documentclass[acmsmall]{acmart}

%%
%% \BibTeX command to typeset BibTeX logo in the docs
\AtBeginDocument{%
  \providecommand\BibTeX{{%
    Bib\TeX}}}

%% Rights management information.  This information is sent to you
%% when you complete the rights form.  These commands have SAMPLE
%% values in them; it is your responsibility as an author to replace
%% the commands and values with those provided to you when you
%% complete the rights form.
\setcopyright{acmcopyright}
\copyrightyear{2018}
\acmYear{2018}
\acmDOI{XXXXXXX.XXXXXXX}


%%
%% These commands are for a JOURNAL article.
\acmJournal{JACM}
\acmVolume{37}
\acmNumber{4}
\acmArticle{111}
\acmMonth{8}

%%
%% Submission ID.
%% Use this when submitting an article to a sponsored event. You'll
%% receive a unique submission ID from the organizers
%% of the event, and this ID should be used as the parameter to this command.
%%\acmSubmissionID{123-A56-BU3}

%%
%% For managing citations, it is recommended to use bibliography
%% files in BibTeX format.
%%
%% You can then either use BibTeX with the ACM-Reference-Format style,
%% or BibLaTeX with the acmnumeric or acmauthoryear sytles, that include
%% support for advanced citation of software artefact from the
%% biblatex-software package, also separately available on CTAN.
%%
%% Look at the sample-*-biblatex.tex files for templates showcasing
%% the biblatex styles.
%%

%%
%% The majority of ACM publications use numbered citations and
%% references.  The command \citestyle{authoryear} switches to the
%% "author year" style.
%%


%%
%% end of the preamble, start of the body of the document source.
\begin{document}

%%
%% The "title" command has an optional parameter,
%% allowing the author to define a "short title" to be used in page headers.
\title{An Update to the Overview of the Trilinos Library}

%%
%% The "author" command and its associated commands are used to define
%% the authors and their affiliations.
%% Of note is the shared affiliation of the first two authors, and the
%% "authornote" and "authornotemark" commands
%% used to denote shared contribution to the research.
\author{Sivasankaran Rajamanickam}
\email{srajama@sandia.gov}
\orcid{1234-5678-9012}
\author{Michael Heroux}
\email{maherou@sandia.gov}
\author{Kim Liegeois}
\email{knliege@sandia.gov}
\orcid{0000-0002-1182-4078}
\author{New Sandia Author}
\email{TBD@sandia.gov}

\affiliation{%
  \institution{Sandia National Laboratories}
  \streetaddress{1515, Eubank Blvd SE}
  \city{Albuquerque}
  \state{New Mexico}
  \country{USA}
  \postcode{87123}
}

\author{Alexander Heinlein}
\affiliation{%
  \institution{TU Delft}
  \streetaddress{TBD}
  \city{TBD}
  \country{Netherlands}}
\email{A.Heinlein@tudelft.nl}

\author{New Non Sandia Author}
\affiliation{%
  \institution{TBD}
  \city{TBD}
  \country{TBD}
}





%%
%% By default, the full list of authors will be used in the page
%% headers. Often, this list is too long, and will overlap
%% other information printed in the page headers. This command allows
%% the author to define a more concise list
%% of authors' names for this purpose.
\renewcommand{\shortauthors}{Trovato et al.}

%%
%% The abstract is a short summary of the work to be presented in the
%% article.
\begin{abstract}
 This is an update to the ``Overview of Trilinos document.'' Trilinos framework has undergone substantial changes to support new applications and new hardware architectures. We describe the new organization, features and design of Trilinos here.
\end{abstract}

%%
%% The code below is generated by the tool at http://dl.acm.org/ccs.cfm.
%% Please copy and paste the code instead of the example below.
%%
%\begin{CCSXML}
%<ccs2012>
% <concept>
%  <concept_id>10010520.10010553.10010562</concept_id>
%  <concept_desc>Computer systems organization~Embedded systems</concept_desc>
%  <concept_significance>500</concept_significance>
% </concept>
% <concept>
%  <concept_id>10010520.10010575.10010755</concept_id>
%  <concept_desc>Computer systems organization~Redundancy</concept_desc>
%  <concept_significance>300</concept_significance>
% </concept>
% <concept>
%  <concept_id>10010520.10010553.10010554</concept_id>
%  <concept_desc>Computer systems organization~Robotics</concept_desc>
%  <concept_significance>100</concept_significance>
% </concept>
% <concept>
%  <concept_id>10003033.10003083.10003095</concept_id>
%  <concept_desc>Networks~Network reliability</concept_desc>
%  <concept_significance>100</concept_significance>
% </concept>
%</ccs2012>
%\end{CCSXML}

%\ccsdesc[500]{Computer systems organization~Embedded systems}
%\ccsdesc[300]{Computer systems organization~Redundancy}
%\ccsdesc{Computer systems organization~Robotics}
%\ccsdesc[100]{Networks~Network reliability}

%%
%% Keywords. The author(s) should pick words that accurately describe
%% the work being presented. Separate the keywords with commas.
\keywords{Scientific Software Frameworks}

\received{15 April 2023}
%\received[revised]{12 March 2009}
%\received[accepted]{5 June 2009}

%%
%% This command processes the author and affiliation and title
%% information and builds the first part of the formatted document.
\maketitle

\section{Introduction}

\section{Data Services}
\todo{Describe Tpetra, idea of performance portability, Kokkos and Kokkos Kernels}

\todo{Do we want Teko here?}

\section{Discretization}
\todo{work on this after the new pruducts structure is defined}
The discretization product contains several packages to handle discretizations of differential equations.

\subsection{Intrepid2}
\todo{Nate, let's coordinate on what to put here}
Intrepid2 provides interoperable tools for compatible discretizations of PDEs. Intrepid2 mainly focus on local assembly of continuous and discontinuous finite elements, and provides tools for finite volume discretizations as well. The present version of Intrepid2 implements compatible finite element spaces of orders up to 10 for H(grad), H(curl), H(div) and L2 function spaces on frequently used elements such as triangles, quadrilaterals, tetrahedrons and hexahedrons. It provides both Lagrangian basis functions and Hierarchical basis functions and it implements sevral performance optimization (sum factorizations) explointing underline structure of the problem (e.g. tensor-product elements or other symmetries). Intrepid2 provides orientation tools for matching the degrees of freedom on shared edges and faces. It also provides projection tools for projecting functions in H(grad), H(curl), H(div) and L2 to the respective discrete spaces. Intrepid2 achieves performance portability using the Kokkos programming model. 

\subsection{Phalanx}
\todo{Roger, please edit/expand}
The package is a local field evaluation kernel specifically designed for general partial differential equation solvers. The main goal of Phalanx is to decompose a complex problem into a number of simpler problems with managed dependencies to support rapid development and extensibility of the PDE code. Through the use of template metaprogramming concepts, Phalanx supports arbitrary user defined data types and evaluation types. This allows for unprecedented flexibility for direct integration with user applications and provides extensive support for embedded technology such as automatic differentiation for sensitivity analysis and uncertainty quantification.

\subsection{Panzer}
\todo{Roger, please edit/expand}
The package provides global tools for finite element analysis. It handles continuous and discontinuous high-order compatible finite elements, as implemented in Intrepid2 on unstructured meshes. Panzer relies on Phalanx to manage with efficiency and flexibility the assembly of complex problems. Panzer also enables the solution of nonlinear problems, by interfacing with several Trlinos linear and nonlinear solvers. It computes derivatives and sensitivities through automatic differentiation (Sacado). It supports both Epetra and Tpetra data structures and achieves performance portability through the Kokkos programming model.

\subsection{Compadre}
\todo{Paul, please edit/expand}
Compadre toolkit provides tools for the approximation of linear operators applied to a function (including point evaluation and derivatives), given samples of the function over a cloud of points. The toolkit can be used for data transfer applications as well as for meshless discretization of PDEs. The package uses generalized moving least squares for approximating functionals and we plan on implementing other meshless methods like radial basis functions. It achieves performance portability by using Kokkos programming model.





\section{Linear Solvers}
\todo{Describe new linear solver packages, ShyLU, FROSch, new packages similar to old packages, new capabilities in these packages (two level DD), GPU supported features}

Trilinos offers several linear solver capabilities that vary from direct solvers, iterative solvers, preconditioners that are local to a compute node to scalable domain decomposition and multigrid methods. Almost all the capabilities described here are focused on the current generation Trilinos using the Tpetra stack. All of these solver capabilities are built on top of Kokkos and are GPU capable to varying degrees. We present any exception to this in the detailed descriptions below.

\subsection{Iterative Linear Solvers}

\subsection{Krylov methods: Belos}
\todo{JHU: it seems strange not to mention Belos. Belos paper: \cite{Bavier2012a}}

\subsection{Domain Decomposition and Basic Iterative Methods: Ifpack2}

Trilinos provides domain decomposition approaches in two different
packages: Ifpack2 and ShyLU (specifically the ShyLU\_DD
subpackage). Ifpack2 implements overlapping additive Schwarz
approaches with several options for the local subdomain solves. The
local subdomain solvers may either be CPU-only versions of incomplete
factorization preconditioners implemented in Ifpack2 itself, such as
ILU(k) and ILUt (thresholded ILU), or architecture portable algorithms
for incomplete factorizations and triangular solvers implemented in
Kokkos Kernels. It is possible to use direct solvers as subdomain
solvers as well. There are options to call shared-memory inexact
incomplete factorization preconditioners in ShyLU as well. One-level
preconditioners such as these are used as smoothers within multigrid
methods, for solving ``simpler'' problems where the setup cost of a
more robust multilevel methods is prohibitive when compared to the
reduction in the number of iterations, or when the underlying problem
is simply not amenable to multigrid methods. 

Ifpack2 also supplies basic iterative methods, such as Jacobi
iteration, Gauss-Seidel, a MPI-oriented hybrid of Jacobi and
Gauss-Seidel (e.g. Jacobi between ranks and Gauss-Seidel on them) and
Chebyshev iteration.  These are available both in point and block
forms and can operate on CSR or BSR matrices.  In the block case,
line relaxation is also supported, while in the point case, techniques
like Vanka relaxation \cite{Vanka1986} can be supported.  Auxiliary
smoothing for $H(curl)$ and $H(div)$ discretizations, of the style of
Hiptmair \cite{Hiptmair1997} are also supported.
Local kernels for these methods are either
implemented in Ifpack2 proper, Tpetra or called from Kokkos kernels.
\todo{Jonathan, Brian: expand / fix?}


\subsection{Multilevel Domain Decomposition Methods: FROSch}
\label{ssec:frosch}

FROSch (Fast and Robust Overlapping Schwarz) is a framework for the construction of multilevel Schwarz domain decomposition solvers. Besides parallel scalability, FROSch focuses on a wide range of applicability and robustness for challenging problems while allowing for an algebraic construction of the Schwarz operators, that is, the construction is only based on the fully assembled system matrix. This is facilitated by an algebraic construction of an overlapping domain decomposition on the first level, as in Ifpack2, as well as the use of extension-based coarse spaces, such as in the classical two-level generalized Dryja--Smith--Widlund (GDSW) preconditioner~\cite{dohrmann_domain_2008} and related variants. While the first version was still based on the Epetra linear algebra framework~\cite{heinlein_parallel_2016}, the current implementation of FROSch is based on Xpetra~\cite{heinlein_frosch_2020}, which allows the use of both Epetra and Tpetra linear algebra stacks. Algorithmic variants of Schwarz methods implemented in FROSch include:
\begin{itemize}
	\item \emph{Extension-based coarse spaces based on a partition of unity on the interface}, such as classical GDSW, reduced dimension GDSW (RGDSW) coarse spaces, and multiscale finite element method (MsFEM) coarse spaces.
	\item \emph{Monolithic Schwarz preconditioners} for block systems.
	\item \emph{Multilevel Schwarz preconditioners}, which are obtained from two-level Schwarz preconditioners by by recursively applying Schwarz preconditioners as an inexact solver for the coarse problems.
\end{itemize}
FROSch has been applied to various challenging application problems, including scalar elliptic and elasticity problems~\cite{heinlein_parallel_2016}, computational fluid dynamics problems~\cite{heinlein_monolithic_2019}, and coupled multiphysics problems for land ice simulations~\cite{heinlein_frosch_2022}; the latter two have been solved using monolithic preconditioning techniques. To obtain robust convergence for heterogeneous model problems, spectral coarse spaces will be implemented shortly~\cite{heinlein_adaptive_2019}. FROSch preconditioners have scaled to more than 200\,k processor cores on the Theta Cray XC40 supercomputer at the Argonne Leadership Computing Facility (ALCF); cf.~\cite{heinlein_parallel_2022}.

In the current implementation, FROSch assumes a one-to-one correspondence of subdomains and MPI ranks, however, due to an interface to the other solver packages in Trilinos, inexact subdomain solvers can be employed on subdomains. An extension to multiple subdomains per MPI rank is currently being implemented. Using Kokkos and KokkosKernels, which are available through the Tpetra linear algebra framework, FROSch has recently also been ported to GPUs~\cite{Yamazaki:2022:EST}, with performance gains for the triangular solve or inexact solves with ILU on the GPUs.

A demo/tutorial for FROSch can be found at the GitHub repository~\cite{frosch_demo}.

\subsection{Multigrid Methods: MueLu}

MueLu is a flexible and scalable high-performance multigrid solver library.
It provides a variety of multigrid algorithms for problems ranging from Poisson-like operators over elasticity, convection-diffusion, and Navier-Stokes, and Maxwell’s equations
all the way to multigrid methods for coupled multiphysics systems.
Besides its strong focus on aggregation-based algebraic multigrid (AMG) methods,
MueLu comes with specialized capabilities for (semi-)structured grids to perform (semi-)coarsening along grid lines,
yet forming the coarse operator via a Galerkin product (in contrast to classical geometric multigrid).
MueLu is extensible and allows for the research and development of new multigrid preconditioning methods.
Its weak and strong scalability even for vector-valued PDEs on unstructured meshes
up to 131,000 cores of a Cray XC40 and one million cores of a Blue Gene/Q system have been shown in~\cite{Lin2017a,Thomas2019a}.
\todo{CG: Find some newer runs.}

\todo{Should we comment on the current state of Kokkos in MueLu? CG: Yes}

MueLu provides several approaches to constructing and solving the multilevel problem:

\begin{itemize}
\item \emph{Algebraic smoothed aggregation approach}~\cite{Vanek1996a}:
The matrix graph is colored to create aggregates (groups) of nodes.
These aggregates define a tentative projection operator.
A final projection operator is created by applying a smoother to the tentative operator.

\item \emph{Algebraic multigrid for Maxwell’s equations}:
\todo{CG: I can write something}

\item \emph{Multigrid for multiphysics}:
MueLu implements a tool box to compile multi-level block preconditioners for block matrices arising from coupled multiphysics problems.
Applications range from Navier-Stokes equations
over surface-coupling (as in fluid/structure interaction or contact mechanics~\cite{Wiesner2021a})
to volume-coupled problems (e.g. in magneto-hydro dynamics~\cite{Ohm2022a}).

\item \emph{Semi-coarsening algebraic multigrid approach}~\cite{Tuminaro2016a}:
Specialized aggregation procedures for three-dimensional meshes generated by extrusion of a two-dimensional unstructured grid
allow to first coarsen in the direction of extrusion to reduce the system to a two-dimensional representation and then perform classical aggregation-based AMG
for the remaining coarsenings.

\item \emph{AMG for (semi-)structured grids}:
Structured aggregation allows to identify coarse points with a user-given coarsening rate and compute interpolation operators.
The coarse operators are then formed via a Galerkin product to avoid remeshing on the coarse levels.
This work has been extended to semi-structured grids to leverage structured-grid computational performance also for globally unstructured grids~\cite{Mayr2022a}.

\end{itemize}

Several resources provide insight into MueLu:
An overview is given on the MueLu website\footnote{\url{https://trilinos.github.io/muelu.html}}.
The MueLu User's Guide~\cite{BergerVergiat2023a} summarizes installation instructions and a reference to most of MueLu's configuration parameters.
The MueLu Tutorial~\cite{Mayr2023b} introduces beginners and experts to various topics in MueLu and shows how to solve or precondition different linear systems using MueLu.
Details on the compatibility of MueLu and its predecessor ML~\cite{Heroux2005a,Gee2006a} can be found in the MueLu User's Guide~\cite{BergerVergiat2023a}.

Besides its C++ API, MueLu offers a MATLAB interface, MueMex, to provide access to MueLu's aggregation and solver routines from MATLAB.
MueMex allows users to setup and solve arbitrarily many problems with either MueLu as a preconditioner, Belos as a solver and Epetra or Tpetra for data structures.


\subsection{Direct Linear Solvers}

\todo{Amesos2~\cite{Bavier2012a}: here or in Section~\ref{sec:TPLDirectSolvers}?}

\subsection{Native Direct / Hybrid Linear Solvers: ShyLU}
 The Schur complement based hybrid solver in Trilinos is implemented in ShyLU. This is a hybrid direct and iterative solver where the subdomains use a direct solver and the Schur complement is solved using an iterative approach. The preconditioner for the Schur complement solver is computed using a probing approach or using a threshold based dropping strategy. This solver was developed to address the requirements of circuit simulation applications. The solver is also hybrid in the parallel computing sense as it uses MPI+threads. This  solver algorithm is not focused on GPU architectures. As a result, we have not implemented this solver using Kokkos. This solver has been shown to be useful for circuit simulation applications.

 ShyLU also has two sparse direct linear solvers. Basker is a sparse LU factorization focused on problems that have the block triangular form structure typically seen in circuit simulation applications. Basker uses these structures to factor and solve the diagonal blocks in parallel. The larger diagonal blocks can themselves be factored in parallel by discovering the parallelism available using a nested-dissection reordering. Basker is focused on the CPU architectures. However, we have still implemented the solver using Kokkos. \todo{Ichi, Nathan expand / add?}

 Tacho computes supernodal sparse Cholesky factorizations. This was designed for problems that are common in the mechanics applications. Tacho exploits the supernodal structure both in factorization and triangular solve phase.  Trilinos also has a templated implementation of the KLU solver called KLU2 to be used as a native coarse solver for multigrid preconditioners.

\subsection{Interfaces to Third-party Direct Linear Solvers}
\label{sec:TPLDirectSolvers}

\subsection{Eigensolvers}
\todo{Do we want that here or somewhere else? SR: Has there been any new development in Eigen solvers recently? Should we just mention they work with new stack?}

\todo{CG: How about Stratimikos or Stratimikos2? }


\section{Nonlinear Solvers}
\todo{Describe new nonlinear solver features}

\todo{Describe new embedded analysis capabilitites, discuss Sacado, Stokhos, Piro, Loca updates}

Trilinos have several embedded analysis capabilities.

\subsection{Piro: the unifying package of the Embedded Nonlinear Analysis Capability area}

\subsection{Sacado}

\subsection{Stokhos}

\subsection{Loca}




\section{Analysis capabilities}
\todo{Describe new embedded analysis capabilitites, discuss Sacado, Stokhos, Piro, Loca updates}

Trilinos have several embedded analysis capabilities.

\subsection{Piro: the unifying package of the Embedded Nonlinear Analysis Capability area}

\subsection{Sacado}

\subsection{Stokhos}

\subsection{Loca}




\section{Trilinos Framework}
\todo{Describe the modern framework, packages like pyTrilinos}
\todo{CG: PyTrilinos has currently no developer.}

Journals use one of three template styles. All but three ACM journals
use the {\verb|acmsmall|} template style:
\begin{itemize}
\item {\texttt{acmsmall}}: The default journal template style.
\item {\texttt{acmlarge}}: Used by JOCCH and TAP.
\item {\texttt{acmtog}}: Used by TOG.
\end{itemize}

The majority of conference proceedings documentation will use the {\verb|acmconf|} template style.
\begin{itemize}
\item {\texttt{acmconf}}: The default proceedings template style.
\item{\texttt{sigchi}}: Used for SIGCHI conference articles.
\item{\texttt{sigplan}}: Used for SIGPLAN conference articles.
\end{itemize}

\subsection{Template Parameters}

In addition to specifying the {\itshape template style} to be used in
formatting your work, there are a number of {\itshape template parameters}
which modify some part of the applied template style. A complete list
of these parameters can be found in the {\itshape \LaTeX\ User's Guide.}

Frequently-used parameters, or combinations of parameters, include:
\begin{itemize}
\item {\texttt{anonymous,review}}: Suitable for a ``double-blind''
  conference submission. Anonymizes the work and includes line
  numbers. Use with the \texttt{\acmSubmissionID} command to print the
  submission's unique ID on each page of the work.
\item{\texttt{authorversion}}: Produces a version of the work suitable
  for posting by the author.
\item{\texttt{screen}}: Produces colored hyperlinks.
\end{itemize}

This document uses the following string as the first command in the
source file:
\begin{verbatim}
\documentclass[acmsmall]{acmart}
\end{verbatim}

\section{Modifications}

Modifying the template --- including but not limited to: adjusting
margins, typeface sizes, line spacing, paragraph and list definitions,
and the use of the \verb|\vspace| command to manually adjust the
vertical spacing between elements of your work --- is not allowed.

{\bfseries Your document will be returned to you for revision if
  modifications are discovered.}

\section{Typefaces}

The ``\verb|acmart|'' document class requires the use of the
``Libertine'' typeface family. Your \TeX\ installation should include
this set of packages. Please do not substitute other typefaces. The
``\verb|lmodern|'' and ``\verb|ltimes|'' packages should not be used,
as they will override the built-in typeface families.

\section{Title Information}

The title of your work should use capital letters appropriately -
\url{https://capitalizemytitle.com/} has useful rules for
capitalization. Use the {\verb|title|} command to define the title of
your work. If your work has a subtitle, define it with the
{\verb|subtitle|} command.  Do not insert line breaks in your title.

If your title is lengthy, you must define a short version to be used
in the page headers, to prevent overlapping text. The \verb|title|
command has a ``short title'' parameter:
\begin{verbatim}
  \title[short title]{full title}
\end{verbatim}

\section{Authors and Affiliations}

Each author must be defined separately for accurate metadata
identification.  As an exception, multiple authors may share one
affiliation. Authors' names should not be abbreviated; use full first
names wherever possible. Include authors' e-mail addresses whenever
possible.

Grouping authors' names or e-mail addresses, or providing an ``e-mail
alias,'' as shown below, is not acceptable:
\begin{verbatim}
  \author{Brooke Aster, David Mehldau}
  \email{dave,judy,steve@university.edu}
  \email{firstname.lastname@phillips.org}
\end{verbatim}

The \verb|authornote| and \verb|authornotemark| commands allow a note
to apply to multiple authors --- for example, if the first two authors
of an article contributed equally to the work.

If your author list is lengthy, you must define a shortened version of
the list of authors to be used in the page headers, to prevent
overlapping text. The following command should be placed just after
the last \verb|\author{}| definition:
\begin{verbatim}
  \renewcommand{\shortauthors}{McCartney, et al.}
\end{verbatim}
Omitting this command will force the use of a concatenated list of all
of the authors' names, which may result in overlapping text in the
page headers.

The article template's documentation, available at
\url{https://www.acm.org/publications/proceedings-template}, has a
complete explanation of these commands and tips for their effective
use.

Note that authors' addresses are mandatory for journal articles.

\section{Rights Information}

Authors of any work published by ACM will need to complete a rights
form. Depending on the kind of work, and the rights management choice
made by the author, this may be copyright transfer, permission,
license, or an OA (open access) agreement.

Regardless of the rights management choice, the author will receive a
copy of the completed rights form once it has been submitted. This
form contains \LaTeX\ commands that must be copied into the source
document. When the document source is compiled, these commands and
their parameters add formatted text to several areas of the final
document:
\begin{itemize}
\item the ``ACM Reference Format'' text on the first page.
\item the ``rights management'' text on the first page.
\item the conference information in the page header(s).
\end{itemize}

Rights information is unique to the work; if you are preparing several
works for an event, make sure to use the correct set of commands with
each of the works.

The ACM Reference Format text is required for all articles over one
page in length, and is optional for one-page articles (abstracts).

\section{CCS Concepts and User-Defined Keywords}

Two elements of the ``acmart'' document class provide powerful
taxonomic tools for you to help readers find your work in an online
search.

The ACM Computing Classification System ---
\url{https://www.acm.org/publications/class-2012} --- is a set of
classifiers and concepts that describe the computing
discipline. Authors can select entries from this classification
system, via \url{https://dl.acm.org/ccs/ccs.cfm}, and generate the
commands to be included in the \LaTeX\ source.

User-defined keywords are a comma-separated list of words and phrases
of the authors' choosing, providing a more flexible way of describing
the research being presented.

CCS concepts and user-defined keywords are required for for all
articles over two pages in length, and are optional for one- and
two-page articles (or abstracts).

\section{Sectioning Commands}

Your work should use standard \LaTeX\ sectioning commands:
\verb|section|, \verb|subsection|, \verb|subsubsection|, and
\verb|paragraph|. They should be numbered; do not remove the numbering
from the commands.

Simulating a sectioning command by setting the first word or words of
a paragraph in boldface or italicized text is {\bfseries not allowed.}

\section{Tables}

The ``\verb|acmart|'' document class includes the ``\verb|booktabs|''
package --- \url{https://ctan.org/pkg/booktabs} --- for preparing
high-quality tables.

Table captions are placed {\itshape above} the table.

Because tables cannot be split across pages, the best placement for
them is typically the top of the page nearest their initial cite.  To
ensure this proper ``floating'' placement of tables, use the
environment \textbf{table} to enclose the table's contents and the
table caption.  The contents of the table itself must go in the
\textbf{tabular} environment, to be aligned properly in rows and
columns, with the desired horizontal and vertical rules.  Again,
detailed instructions on \textbf{tabular} material are found in the
\textit{\LaTeX\ User's Guide}.

Immediately following this sentence is the point at which
Table~\ref{tab:freq} is included in the input file; compare the
placement of the table here with the table in the printed output of
this document.

\begin{table}
  \caption{Frequency of Special Characters}
  \label{tab:freq}
  \begin{tabular}{ccl}
    \toprule
    Non-English or Math&Frequency&Comments\\
    \midrule
    \O & 1 in 1,000& For Swedish names\\
    $\pi$ & 1 in 5& Common in math\\
    \$ & 4 in 5 & Used in business\\
    $\Psi^2_1$ & 1 in 40,000& Unexplained usage\\
  \bottomrule
\end{tabular}
\end{table}

To set a wider table, which takes up the whole width of the page's
live area, use the environment \textbf{table*} to enclose the table's
contents and the table caption.  As with a single-column table, this
wide table will ``float'' to a location deemed more
desirable. Immediately following this sentence is the point at which
Table~\ref{tab:commands} is included in the input file; again, it is
instructive to compare the placement of the table here with the table
in the printed output of this document.

\begin{table*}
  \caption{Some Typical Commands}
  \label{tab:commands}
  \begin{tabular}{ccl}
    \toprule
    Command &A Number & Comments\\
    \midrule
    \texttt{{\char'134}author} & 100& Author \\
    \texttt{{\char'134}table}& 300 & For tables\\
    \texttt{{\char'134}table*}& 400& For wider tables\\
    \bottomrule
  \end{tabular}
\end{table*}

Always use midrule to separate table header rows from data rows, and
use it only for this purpose. This enables assistive technologies to
recognise table headers and support their users in navigating tables
more easily.

\section{Math Equations}
You may want to display math equations in three distinct styles:
inline, numbered or non-numbered display.  Each of the three are
discussed in the next sections.

\subsection{Inline (In-text) Equations}
A formula that appears in the running text is called an inline or
in-text formula.  It is produced by the \textbf{math} environment,
which can be invoked with the usual
\texttt{{\char'134}begin\,\ldots{\char'134}end} construction or with
the short form \texttt{\$\,\ldots\$}. You can use any of the symbols
and structures, from $\alpha$ to $\omega$, available in
\LaTeX~\cite{Lamport:LaTeX}; this section will simply show a few
examples of in-text equations in context. Notice how this equation:
\begin{math}
  \lim_{n\rightarrow \infty}x=0
\end{math},
set here in in-line math style, looks slightly different when
set in display style.  (See next section).

\subsection{Display Equations}
A numbered display equation---one set off by vertical space from the
text and centered horizontally---is produced by the \textbf{equation}
environment. An unnumbered display equation is produced by the
\textbf{displaymath} environment.

Again, in either environment, you can use any of the symbols and
structures available in \LaTeX\@; this section will just give a couple
of examples of display equations in context.  First, consider the
equation, shown as an inline equation above:
\begin{equation}
  \lim_{n\rightarrow \infty}x=0
\end{equation}
Notice how it is formatted somewhat differently in
the \textbf{displaymath}
environment.  Now, we'll enter an unnumbered equation:
\begin{displaymath}
  \sum_{i=0}^{\infty} x + 1
\end{displaymath}
and follow it with another numbered equation:
\begin{equation}
  \sum_{i=0}^{\infty}x_i=\int_{0}^{\pi+2} f
\end{equation}
just to demonstrate \LaTeX's able handling of numbering.

\section{Figures}

The ``\verb|figure|'' environment should be used for figures. One or
more images can be placed within a figure. If your figure contains
third-party material, you must clearly identify it as such, as shown
in the example below.
%\begin{figure}[h]
%  \centering
%  \includegraphics[width=\linewidth]{sample-franklin}
%  \caption{1907 Franklin Model D roadster. Photograph by Harris \&
%    Ewing, Inc. [Public domain], via Wikimedia
%    Commons. (\url{https://goo.gl/VLCRBB}).}
%  \Description{A woman and a girl in white dresses sit in an open car.}
%\end{figure}

Your figures should contain a caption which describes the figure to
the reader.

Figure captions are placed {\itshape below} the figure.

Every figure should also have a figure description unless it is purely
decorative. These descriptions convey what’s in the image to someone
who cannot see it. They are also used by search engine crawlers for
indexing images, and when images cannot be loaded.

A figure description must be unformatted plain text less than 2000
characters long (including spaces).  {\bfseries Figure descriptions
  should not repeat the figure caption – their purpose is to capture
  important information that is not already provided in the caption or
  the main text of the paper.} For figures that convey important and
complex new information, a short text description may not be
adequate. More complex alternative descriptions can be placed in an
appendix and referenced in a short figure description. For example,
provide a data table capturing the information in a bar chart, or a
structured list representing a graph.  For additional information
regarding how best to write figure descriptions and why doing this is
so important, please see
\url{https://www.acm.org/publications/taps/describing-figures/}.

\subsection{The ``Teaser Figure''}

A ``teaser figure'' is an image, or set of images in one figure, that
are placed after all author and affiliation information, and before
the body of the article, spanning the page. If you wish to have such a
figure in your article, place the command immediately before the
\verb|\maketitle| command:
\begin{verbatim}
  \begin{teaserfigure}
    \includegraphics[width=\textwidth]{sampleteaser}
    \caption{figure caption}
    \Description{figure description}
  \end{teaserfigure}
\end{verbatim}

\section{Citations and Bibliographies}

The use of \BibTeX\ for the preparation and formatting of one's
references is strongly recommended. Authors' names should be complete
--- use full first names (``Donald E. Knuth'') not initials
(``D. E. Knuth'') --- and the salient identifying features of a
reference should be included: title, year, volume, number, pages,
article DOI, etc.

The bibliography is included in your source document with these two
commands, placed just before the \verb|\end{document}| command:
\begin{verbatim}
  \bibliographystyle{ACM-Reference-Format}
  \bibliography{bibfile}
\end{verbatim}
where ``\verb|bibfile|'' is the name, without the ``\verb|.bib|''
suffix, of the \BibTeX\ file.

Citations and references are numbered by default. A small number of
ACM publications have citations and references formatted in the
``author year'' style; for these exceptions, please include this
command in the {\bfseries preamble} (before the command
``\verb|\begin{document}|'') of your \LaTeX\ source:
\begin{verbatim}
  \citestyle{acmauthoryear}
\end{verbatim}


  Some examples.  A paginated journal article \cite{Abril07}, an
  enumerated journal article \cite{Cohen07}, a reference to an entire
  issue \cite{JCohen96}, a monograph (whole book) \cite{Kosiur01}, a
  monograph/whole book in a series (see 2a in spec. document)
  \cite{Harel79}, a divisible-book such as an anthology or compilation
  \cite{Editor00} followed by the same example, however we only output
  the series if the volume number is given \cite{Editor00a} (so
  Editor00a's series should NOT be present since it has no vol. no.),
  a chapter in a divisible book \cite{Spector90}, a chapter in a
  divisible book in a series \cite{Douglass98}, a multi-volume work as
  book \cite{Knuth97}, a couple of articles in a proceedings (of a
  conference, symposium, workshop for example) (paginated proceedings
  article) \cite{Andler79, Hagerup1993}, a proceedings article with
  all possible elements \cite{Smith10}, an example of an enumerated
  proceedings article \cite{VanGundy07}, an informally published work
  \cite{Harel78}, a couple of preprints \cite{Bornmann2019,
    AnzarootPBM14}, a doctoral dissertation \cite{Clarkson85}, a
  master's thesis: \cite{anisi03}, an online document / world wide web
  resource \cite{Thornburg01, Ablamowicz07, Poker06}, a video game
  (Case 1) \cite{Obama08} and (Case 2) \cite{Novak03} and \cite{Lee05}
  and (Case 3) a patent \cite{JoeScientist001}, work accepted for
  publication \cite{rous08}, 'YYYYb'-test for prolific author
  \cite{SaeediMEJ10} and \cite{SaeediJETC10}. Other cites might
  contain 'duplicate' DOI and URLs (some SIAM articles)
  \cite{Kirschmer:2010:AEI:1958016.1958018}. Boris / Barbara Beeton:
  multi-volume works as books \cite{MR781536} and \cite{MR781537}. A
  couple of citations with DOIs:
  \cite{2004:ITE:1009386.1010128,Kirschmer:2010:AEI:1958016.1958018}. Online
  citations: \cite{TUGInstmem, Thornburg01, CTANacmart}.
  Artifacts: \cite{R} and \cite{UMassCitations}.

\section{Acknowledgments}

Identification of funding sources and other support, and thanks to
individuals and groups that assisted in the research and the
preparation of the work should be included in an acknowledgment
section, which is placed just before the reference section in your
document.

This section has a special environment:
\begin{verbatim}
  \begin{acks}
  ...
  \end{acks}
\end{verbatim}
so that the information contained therein can be more easily collected
during the article metadata extraction phase, and to ensure
consistency in the spelling of the section heading.

Authors should not prepare this section as a numbered or unnumbered {\verb|\section|}; please use the ``{\verb|acks|}'' environment.

\section{Appendices}

If your work needs an appendix, add it before the
``\verb|\end{document}|'' command at the conclusion of your source
document.

Start the appendix with the ``\verb|appendix|'' command:
\begin{verbatim}
  \appendix
\end{verbatim}
and note that in the appendix, sections are lettered, not
numbered. This document has two appendices, demonstrating the section
and subsection identification method.

\section{Multi-language papers}

Papers may be written in languages other than English or include
titles, subtitles, keywords and abstracts in different languages (as a
rule, a paper in a language other than English should include an
English title and an English abstract).  Use \verb|language=...| for
every language used in the paper.  The last language indicated is the
main language of the paper.  For example, a French paper with
additional titles and abstracts in English and German may start with
the following command
\begin{verbatim}
\documentclass[sigconf, language=english, language=german,
               language=french]{acmart}
\end{verbatim}

The title, subtitle, keywords and abstract will be typeset in the main
language of the paper.  The commands \verb|\translatedXXX|, \verb|XXX|
begin title, subtitle and keywords, can be used to set these elements
in the other languages.  The environment \verb|translatedabstract| is
used to set the translation of the abstract.  These commands and
environment have a mandatory first argument: the language of the
second argument.  See \verb|sample-sigconf-i13n.tex| file for examples
of their usage.

\section{SIGCHI Extended Abstracts}

The ``\verb|sigchi-a|'' template style (available only in \LaTeX\ and
not in Word) produces a landscape-orientation formatted article, with
a wide left margin. Three environments are available for use with the
``\verb|sigchi-a|'' template style, and produce formatted output in
the margin:
\begin{description}
\item[\texttt{sidebar}:]  Place formatted text in the margin.
\item[\texttt{marginfigure}:] Place a figure in the margin.
\item[\texttt{margintable}:] Place a table in the margin.
\end{description}

%%
%% The acknowledgments section is defined using the "acks" environment
%% (and NOT an unnumbered section). This ensures the proper
%% identification of the section in the article metadata, and the
%% consistent spelling of the heading.
\begin{acks}
To Robert, for the bagels and explaining CMYK and color spaces.
\end{acks}

%%
%% The next two lines define the bibliography style to be used, and
%% the bibliography file.
\bibliographystyle{ACM-Reference-Format}
\bibliography{sample-base}


%%
%% If your work has an appendix, this is the place to put it.
\appendix

\section{Research Methods}

\subsection{Part One}

Lorem ipsum dolor sit amet, consectetur adipiscing elit. Morbi
malesuada, quam in pulvinar varius, metus nunc fermentum urna, id
sollicitudin purus odio sit amet enim. Aliquam ullamcorper eu ipsum
vel mollis. Curabitur quis dictum nisl. Phasellus vel semper risus, et
lacinia dolor. Integer ultricies commodo sem nec semper.

\subsection{Part Two}

Etiam commodo feugiat nisl pulvinar pellentesque. Etiam auctor sodales
ligula, non varius nibh pulvinar semper. Suspendisse nec lectus non
ipsum convallis congue hendrerit vitae sapien. Donec at laoreet
eros. Vivamus non purus placerat, scelerisque diam eu, cursus
ante. Etiam aliquam tortor auctor efficitur mattis.

\section{Online Resources}

Nam id fermentum dui. Suspendisse sagittis tortor a nulla mollis, in
pulvinar ex pretium. Sed interdum orci quis metus euismod, et sagittis
enim maximus. Vestibulum gravida massa ut felis suscipit
congue. Quisque mattis elit a risus ultrices commodo venenatis eget
dui. Etiam sagittis eleifend elementum.

Nam interdum magna at lectus dignissim, ac dignissim lorem
rhoncus. Maecenas eu arcu ac neque placerat aliquam. Nunc pulvinar
massa et mattis lacinia.

\end{document}
\endinput
%%
%% End of file `sample-acmsmall.tex'.
